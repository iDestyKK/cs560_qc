\indent Generally, the implementation of a HTTP server is based on socket programming in this assignment. First, the code needs to allocate a server socket, bind it to a port, and then listen for incoming connections. Next, the server accepts an incoming client connection and parse the input data stream into a HTTP request. Based on the request's parameters, it then forms a response and sends it back to the client. In a loop, the server perform steps 2 and 3 for as long as the server is running. \\
To observe what a web browser such as Firefox or chrome would send to a HTTP server, you can use netcat to simulate a server. For instance, run the fake server with nc -l -p 10010 and then use your web browser to go to http://hostname:10010. You should then see all HTTP request and headers the web browser sends to the HTTP server. For example, you may see something like following (using Chrome as the browser): \\
\begin{verbatim}
Connection: keep-alive
User-Agent: Mozilla/5.0 (Windows NT 10.0; Win64; x64) 
AppleWebKit/537.36 (KHTML, like Gecko) Chrome/63.0.3239.132 Safari/537.36
Upgrade-Insecure-Requests: 1
Accept: text/html,application/xhtml+xml,application
      /xml;q=0.9,image/webp,image/apng,*/*;q=0.8
Accept-Encoding: gzip, deflate, br
Accept-Language: en-US;,en;q=0.9
\end{verbatim}
In general, a valid HTTP response looks like this:
\begin{verbatim}
HTTP/1.0 200 OK
Content-Type: text/html
<html>
...
</html> 
\end{verbatim}\\
The first line contains the HTTP status code for the request. The second line instructs the client (i.e. web browser) what type of data to expect (i.e. mime-type). Each of these lines should be ended with \r\n. Additionally, it is important that after the Content-Type line you include a blank line consisting of only \r\n. Most web clients will expect this blank line before parsing for the actual content. \\
Once you have implemented a webserver and verified it is working by accessing web pages using a web browser such as Firefox or Chrome, and by using command line tools such as curl or wget. For testing needs, we have provided three HTML pages as samples, namely, index.html, page1.html, and page2.html for your needs. If you place these three pages in a home directory, your server will be able to serve the index.html page by default. This page contains links that link to the second and third page, which you can use as testing benchmarks for your webserver. \\
This programming assignment does not specify any programming language you choose. You may choose any mainstream programming language such as Java, Python, or C to implement the requirements. You do, however, are required to provide detailed testing results in the form of screenshots on how to run your code, what results you obtain, and how you tested your results to know that they are correct. \\
We assume that if there is a page named "index.html" in your home folder, then it will be served as the default page. Besides static HTML pages, you need to study the HTTP protocol, such as the one available here: \\
\textbf{https://www.w3.org/Protocols/HTTP/1.1/rfc2616bis/draft-lafon-rfc2616bis-03.html} \\
So that you will only allow users to submit a basic HTTP form. This is usually done by using a SUBMIT button on a web page, and the web server will record the results into a local file. The details on the implementation will depend on your design choices. You need to document your design choices in detail, and provide necessary screenshots to illustrate your results. \\
\textbf{Note that, prepackaged HTTP server classes, such as Python http.server class, should not be used for this project. Usually these classes provide much more functions than what is required here, and a project that directly calls their APIs will not be given project credits.} 
