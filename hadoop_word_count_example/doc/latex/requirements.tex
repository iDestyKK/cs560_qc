\subsection{Identifying and removing stop words}
One issue is that some words are so common that their presence in an inverted index is"noise," that is they can obfuscate the more interesting properties of a document. Such words are called “stop words.” For this part of the assignment, write a word count Map Reduce function to perform a word count over a corpus of text files and to identify stop words. It is up to you to choose a reasonable threshold (word count frequency) for stop words, but make sure you provide adequate justification and explanation of your choice. A parser will group words by attributes which are not relevant to their meaning (e.g.,"hello","Hello",and "HELLO" are all the same word), so it is up to you to  define "scrub" however you wish; some suggestions include case-insensitivity,etc. It is not required that you treat “run” and “ran” as the same word, but your parser should handle case insensitivity. Once you have written your code, then run your code and collect the word counts for submission with all your Mapper and Reducer files. 
\subsection{Building the Inverted Index}
For this portion of the assignment, you  will design a MapReduce-based algorithm to calculate the inverted index. To this end, you are to create a full inverted index, which maps words to their documentID  + line number in the document. Note that your final inverted index should not contain the words identified in Step 1. The format of your MapReduce output (i.e.,the inverted index) must be simple enough to be machine-parseable; it is not impossible to  imagine your index  being one of many data structures used  in a search engine's indexing pipeline. Your submitted indexer should be  able to run successfully on one or multiple input txt files, where "successfully" means it should run to completion without errors or exceptions, and generate the correct word->DocID mapping. You are required to submit all relevant Mapper and Reducer Java files, in addition to any supporting code or utilities. 
\subsection{Query the Inverted Index}
Write a query program on top of your full inverted file index that accepts a user-specified query (one or more words)and returns not only the documentIDs but also the locations in the form of line numbers. The query program can be local: it does not need to handle the task using Map-Reduce framework again. It is not required that your query program to return text snippets from the original text files.